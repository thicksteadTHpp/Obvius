Introduction: An overview of OBVIUS

OBVIUS (Object Based Vision and Image Understanding System) is a
package for working with images.  It is being developed by the Vision
Science group at the MIT Media Lab, for research in image processing,
machine vision, and models for human vision.  It is designed with the
following buzz-words in mind: interactive, object-oriented,
machine-independent, user-friendly, extensible.

OBVIUS runs under Common LISP.  Currently we are running in Lucid Common Lisp
on Sun 3's and HP 300 workstations, and in Symbolics Common Lisp on
Symbolics Lisp Machines.

We have developed OBVIUS for our own use, but hope that others may
find it useful as well.  OBVIUS is available, WITHOUT SUPPORT OR
WARRANTY, to researchers in the academic vision community.  OBVIUS is
by no means a polished product; it is somewhat buggy, and future versions
will not necessarily be compatible with the current version. MIT holds
rights to the code, and it is possible that at some future date MIT
will wish to impose licensing fees on a future version.


The goals of OBVIUS.

OBVIUS is designed with a number of goals in mind.  The first is to
make interactive image hacking as convenient as possible.  The second
is to allow for growth, as new functions are developed and
incorporated into OBVIUS packages.  The third is to prevent duplicated
effort by encouraging users to build tools that can be shared in
a standard context.

There are certain tasks that always accompany research with images.
Images must be created, modified, stored, retrieved, manipulated,
and displayed.  This involves a lot of unpleasant system-dependent
code, which most researchers do not wish to deal with.  OBVIUS
provides a set of standard routines to take care of these things.

But OBVIUS in not merely a collection of helpful utilities.  It is an
extension of Common Lisp that allows images (and other objects)
and image operations to be integrated into the programming environment
in a natural way.  OBVIUS also provides a convenient user interface,
with basic support for windows and a mouse (although it does not currently
have a thoroughly mouse/menu-driven interface like a Macintosh).  When
the X window standard settles down we will use it; for now we are using
Suntools on the Suns, ***, etc.

Common Lisp was chosen for a number of reasons.  (1) Lisp is
interactive, so that you can easily experiment with new ideas and see
the results.  (2) Common Lisp is a standard that is available on many
machines.  (3) Common Lisp provides a standard syntax that can be
readily extended to image operations.  Thus we can describe image
operations without inventing a new language, with a new syntax, and a
new parser; Lisp takes care of it all.  (4) An object oriented
standard, called CLOS, will soon be established for Common Lisp.  We
currently use PCL, which is virtually the same, and the transition to
CLOS should be painless. (5) It is easy to extend the environment by
adding new operations and new image types. (6) Lisp environments
traditionally have provided good tools for program development, to
make efficient use of a programmer's time.

There are certain disadvantages to using Lisp, although they are not
as bad as many C/Unix hackers would have you believe.  Lisp requires a
lot of space: to run OBVIUS you need at least 8 megabytes of RAM and 30
megabytes of swap space on your disc.  Lisp code tends to run slower
than C code, although not by much; in our experience the Lucid Lisp
compiler produces code that runs from ** to ** slower than equivalent
C code.  If you demand maximum speed, you can write intensive routines
in C and call them from Lisp; this is what we have done for operations
such as convolution.

Some of the ideas in OBVIUS are derived from the Image-calc [sp?]
package written by Lynn Quam at SRI.




The basics of OBVIUS:

You run OBVIUS from within a Lisp listener.
The preferred editor is EMACS, since we have added a lot of
special patches to EMACS that make it do wonderful things with
OBVIUS. Once you have opened an EMACS buffer, you can start OBVIUS by
saying,

M-x run-obvius

This starts up a Lisp listener, loads in the OBVIUS files, and
sets up enough swap space for image processing.
OBVIUS gives you a prompt, ``OBVIUS>'',
and waits for you to input Lisp functions.  You can do anything
you would normally do in Lisp, such as adding two numbers:

OBVIUS> (+ 10 20)
30
OBVIUS>

You can also do image operations, such as adding two existing images,
using the ``+.'' operator:

OBVIUS> (+. image1 image2)
[***what comes here?]
OBVIUS>

OBVIUS adds the two images, and puts the result in a new image.

Typically you will want to give a name to the resulting image,
which you can do as follows:

OBVIUS>(+. image1 image2 :-> 'image3)
[***]

OBVIUS provides functions to create,
save, load, manipulate, display, and make measurements on images.
Actually, an image is just a special case of a more general object
called a ``viewable,'' which will be explained later; we will refer to
images to simplify the discussion for the moment.  Here are some
examples:


1. Creating images.

OBVIUS provides a set of functions to synthesize images.

OBVIUS> (make-uniform-random-image '(128 128) :-> 'my-noise)
#<IMAGE MY-NOISE (128 128)>
OBVIUS>

makes a new image, 128 x 128 pixels, consisting of random noise drawn
from a uniform distribution, and names it ``my-noise'', and returns a
description of the new object. 



2. Saving images to disc.

You can save an image with by using the function ``save-image'',
as shown here:

[ ** this should be made to work **
(save-image my-noise)

This will save the image to the default directory [??], which is normally [??],
and will give the file the same name as the image. **]

OBVIUS> (save-image my-noise ``/usr/smith/images'') [***???]

will save it to the file specified by the pathname inside the double quotes.
[** must make this more convenient: have a thing like C-c l, that
prompts you for completions.  Even better, do it mac-like.**]

3. Loading images from disc.

To load an image from the disc, say

OBVIUS> (load-image ``/images/my-image'')
#<IMAGE MY-IMAGE (128 128)>

This looks for the image in the file specified by the double-quoted
pathname.  The image name defaults to the
filename.  To load it into an image with a different name, say

OBVIUS> (load-image ``/images/my-image'' :-> 'new-image)
#<IMAGE NEW-IMAGE (128 128)>

A more convenient way to load images is to use the EMACS key sequence
C-c l.  EMACS will prompt you for a pathname, and will offer possible
completions if you type <space> or <tab>. [??**]



4. Viewing images.

The pixel values of an image reside in an array; in order to see them
you must display them.  Image data can be displayed in many ways.  For
instance, it can be printed out in numeric format, it can be graphed
as a surface plot, or it can be shown as a greyscale ``picture'' on a
CRT.  In OBVIUS terminology, a picture is not the same as an image --
a picture is merely a particular way of displaying the information contained
within an image, while the image is the object that contains the 
actual information.

Greyscale pictures are displayed in special windows called ``panes''.
Panes hold circular stacks of pictures.  A new pane may be created by saying:

OBVIUS> (new-pane)
#<OBVIUS::WINDOW 145035663>

A new pane pops onto the screen, and its [descriptor] is returned.

At any given time, one particular pane is the active, or ``current'' pane, and
you can place a picture on it by saying

OBVIUS> (display my-image)
T

In practice, images are usually displayed automatically whenever they
are loaded or modified, so you will rarely have to issue a display command.
In addition, you can see an image by simply evaluating it:

OBVIUS> my-image
#<IMAGE MY-IMAGE (128 128)>

The resulting picture is automatically displayed (unless you have
turned off the auto-display function).


5: Manipulating images.

In Lisp, you can add two variables and put the result in a third
variable.  If a = 3 and b = 4, then you can say:

OBVIUS> (setq c (+ a b))
7

When working with images, the same sorts of operations can be performed
but with a slightly different syntax, as was noted above:

OBVIUS> (+. im1 im2 :-> 'im3)
#<IMAGE IM3 (128 128)>

[Footnote: The reason for this syntax is to give the user better
control over memory allocation than would occur with setq.  In Common
Lisp, every time you call setq you create an entirely new data object,
even if you keep using the same name.  That is, the memory allocated
by (setq x 1) is different from the memory allocated for (setq x 2),
or, for that matter, a second call of (setq x 1).  The old memory
formerly bound to x is abandoned after each new (setq x ...), and
eventually it must be retrieved through garbage collection.  Because
images take up so much space, this would be intolerable when working
with images.]

This creates a new object of the appropriate class, names it im3, and
fills it with the sum.
If an image named im3 already exists, the old image will be renamed ``nil''
[**how about renaming it #im3, ##im3, etc**??].
To save space and reduce the need for garbage collection, you can
destructively modify an existing image by leaving off the quote:

OBVIUS> (+. im1 im2 :-> im3)
#<IMAGE IM3 (128 128)>

This will re-use to old space assigned to im3.

If you leave do not specify a location for the result, OBVIUS will create
an image named ``nil'', and will return its descriptor as ``unnamed''.

OBVIUS> (+. im1 im2)
#<IMAGE Unnamed (128 128)>

You can also also do operations between images and scalars.  For
instance, to double the amplitude of an image, just say

OBVIUS> (*. my-image 2.0 :-> my-image)
#<IMAGE MYNOISE (128 128)>

All of the standard Lisp arithmetic functions will work on images;
the OBVIUS versions of the functions are followed by a dot (.) to
distinguish them.  Examples include: 

+.	-.	*.	/. [...???]

At present, all OBVIUS greyscale images are floating point, i.e., the
pixels are 32 bit FLOATs, and all image arithmetic is floating point
as well.  This slows down the computation and uses up extra space,
and so many users may ask why we do not support  8 or 16-bit integer
image processing.  It is our experience that integer image
processing tends to introduce many subtle and devious errors
into the research process, and that the efficiencies are outweighed by
the dangers.  Nonetheless we will allow integer processing in
a future version of OBVIUS.

Arithmetic comparisons work as you would expect.  For instance,
to do a binary thresholding at the pixel value 10.0, you can say:

OBVIUS> (>. image1 10.0 :-> 'image2)
[***]

And to find all the points where two images are equal, say:

OBVIUS> (=. image1 image2 :-> 'image3)
[***]

These operations return ``bit-images,'' which have values 0 and 1.[??]
You can do Boolean operations on bit images.  Thus, to pick out the
pixels that have grey levels between 10.0 and 20.0, you can say:

OBVIUS> (and. (>. image1 10.0) (<. image1 20.0) :-> 'threshimage)
[***]


6. Making measurements on images.

To find the variance of an image, say:

OBVIUS> (variance my-image)
0.334588

The variance is returned.
The same approach works for mean, min, max,  and other image statistics.
To see a histogram of an image, say

OBVIUS> (make-histogram my-image)
#<HISTOGRAM Histogram of MYNOISE>

and OBVIUS will compute the histogram, draw a picture of it, and
display it in the current pane.  A histogram, by the way, is another
object of the type viewable, just as an image is.  Unlike an image,
however, a histogram's default view-method is [***].
[** Note: I would like to be able to give histograms names just like images:
(make-histogram my-image :-> 'hist1) **]


7. Extending OBVIUS.

It is straightforward to define new functions in OBVIUS.
Since OBVIUS works with objects, these are called methods,
and they are defined to work in different ways with different
object classes.  To understand how to define methods, you
need to know something about the object system (PCL), but
the following example will give the idea.

Suppose we want to define a method called ``double'', that multiplies
the value of all pixels in an image by 2. Then we can write:

OBVIUS> (defmethod double ((im image) &key ->)
	(with-result-image ((result ->) im **??)  [**??]
	(*. im 2.0)))
[**returned value]

[** would like to have a macro that does:
(defmethod double ((im image)) (*. im 2.0))  **]

The method is now defined for images.  To use it for other
viewables, it will be necessary to write method for each of the
other viewables.

It is also possible to add new objects [** how do you do this? **]


8. The user interface.

OBVIUS is designed to run in an interactive environment with windows
and a mouse.  The details depend on the machine that you are using.
Our primary development is on Sun 3's running Unix ** and Suntools **.
We strongly recommend using the
EMACS editor; OBVIUS includes a special package of EMACS
functions that are specifically designed to enhance its utility
for working in OBVIUS.

You will typically have several windows open at a given time.
The EMACS window will contain a Lisp process running OBVIUS,
and will usually contain a scratch buffer in which you are developing code.
The scratch buffer should be in ``Lisp'' mode (not Lisp-interaction mode),
which you can achieve by saying M-x lisp-mode.

You will normally have a few panes open for displaying pictures.
By using the mouse and clicking on panes you can perform many common
operations.  The mouse bindings will
appear in the EMACS minibuffer whenever you put the mouse inside a pane.

We refer to the mouse buttons as LEFT, MIDDLE, and RIGHT; these can be
clicked in conjunction with the special keys CONTROL (C), SHIFT (S)
and META (M). (Unfortunately, on the Sun, the META key is labeled
``Left''.  Try not to be confused.

If you want to cycle through the stack of pictures in a pane,
successively click C-LEFT.  To cycle the other way, click C-MIDDLE
To get a histogram of an image, click on its pane with M-RIGHT.
To destroy an image, hold down the CONTROL, META, and SHIFT
buttons at the same time, and click LEFT; i.e. click C-S-M-LEFT.

Clicking LEFT on a pane selects an viewable, which means that it echos
the name of the viewable into the OBVIUS Lisp buffer.  Therefore
you can enter an image name into ..***


USER'S MANUAL

Vocabulary: get rid of ``picture'' -- it carries too much baggage.
Replace with ``view'' or ``vista'' or ``disp-obj'', or ``pic-obj''.
Or ``rendering'' -- perhaps the most expressive.
``Viewable'' is perhaps funny too.  Use ``im-obj'' or ``im-thing''.

0. The basic structure.
	Variables in Lisp.  Functions in Lisp.  Objects in Lisp.  Methods
in Lisp.
	Objects in OBVIUS.  Methods in OBVIUS.  

1. Loading viewables.  	
	Should be able to say (setqv view-name ``pathname'') 	
	or ``pathname'' :-> viewname.  Shouldn't have to say
	load-image-sequence, etc.

2. Storing/saving viewables.
	Should be able to say (store view-name) and have it default to
	home directory or home image directory.

3. Generating viewables.
	All generating functions start with ``gen''? or ``make'' or ``create''
	Default to *default-size*. specify keyword: :x-size, :y-size.
	All are specified in pixel coordinates.  With 0,0 at middle or not?
	All produce float images as defaults, but can be int or bit w/keys.
	Can you gen a viewable that isn't an image? or only images?

	(gen-line [the equation of the line]) anti-aliased?
	(gen-edge [eqn of edge]) anti-aliased?
	(gen-rect [coords])
	(gen-zone-plate [specs])
	(gen-pinwheel [specs])
	(gen-bulls-eye [inverse zone plate])
	(gen-disc [radius, center])
	(gen-gauss [gaussian blob])
	(gen-uni-noise [seed, range])
	(gen-gau-noise [seed, variance])
	(gen-rnd-dots [density]) -- note-- makes floats)
	(gen-sqr-grat [ori, freq, phase])
	(gen-sin-grat)
	(gen-tri-grat)?
	(gen-saw-grat)?
	(gen-general [function])
	(gen-zero) makes an image full of zeros.
	(gen-nil) makes an image full of nils

Drawing operations: these add information to a currently existing images.
	Added info is normally float.  Lines, etc., can be anti-aliased
	or not. (draw = normal; draaw = anti-aliased?  or keyword)
	Info can be added in several ways:
		:max -- replaces val w/new val if new is greater. like OR
		:min -- replaces with the lesser. Like AND
		:?? how to get something like EXOR?
		:add -- this is the normal method
		:mul -- like filter densities
		:replace -- replace old pixels with non-nil new ones.
	Operations:
	(draw-line)
	(draw-edge) or (draw-step)
	(draw-rect)
	(draw-disc)
	(draw-font)
	(draw-sprite)
	(draw-uni-noise)
	(draw-image)  this does a blit to a portion of existing image;
		Blitted image should be smaller.
	(?? allow one to draw all of the things that one can gen?)

4. Point operations on pixels:
	abs.
	signum.
	sin. cos. tan. asin. acos. atan.
	
	General operation: (general-pointop. [function])
	Defining new point-op: 
		(defun-pointop name (general-pointop. [function]))	

4a. Operations on complex images
	conjugate.
	phase.


5. Binary operations:

	+.  -.  *.  /. 

	=.  /=.  <.  >.  <=.  >=. [Comparisons lead to floats unless
bit is specified? 
 	mag.  	phase.
	
6. Ternary operations:
	(matte. im1 im2 mask-image)

7. Image statistics:
	mean
	variance
	min, max, median, mode
	Also predicates: pyramidp, floatimagep, x-size, y-size

8. Local operations (Neighborhood ops)
	Convolutions
	Correlations
	Median filter
	General local filter (define what happens in neighborhood).
	Space variant filter?
	Local-ops usually have kernels.  For linear convolutions, the kernels
	are the weights.  For median filters, the kernels are the
	binary masks (which also can include weights).

	Is it possible to define predicates like right-edgep, up-right-cornerp,
	so that it is convenient to define new local-ops that do edge
	handling right?

8. General image modifiers:
	(some of these are really point operations)
	clip
	contrast-stretch
	hist-equalize
	gamma-correct
	local-agc
	warp
	resample
	subsample
	type-convert

9. Operations on multiple images (e.g. complex, RGB, volumes, sequences)

10. Pyramid operations.

11. Sounds and other 1-D signals
	We should be able to work with sounds in the same way that we
	work with images.  It should be possible to assemble sounds
	in an object-based way, so that the component sounds maintain
	their identities, and are mixed for purposes of display (playing).
	Display methods for sounds:
	0. Play it as a sound.
	1. Show waveform.  Dots or connect dots.
	2. Zoom in an out; select sub-piece.
	3. Spectrogram: it is an image, which can be shown as
	grey-scale, surface plot, etc. 
Note: sounds may be thought of as extending infinitely in both directions
of time.  Same for image sequences.  Same for spectrograms.  Does this
demand any thought?
	Spectrogram can be short-time, long-time, or self-similar.
	4. It should be possible to line up many parallel tracks, and
	look at them all.  It should be possible to group them, and
	name them.  It should be possible to display sound phrases as
	rectangles (length = duration), with names, so you can assemble
	a complex sound by moving around these labeled blocks.
	5. You should be able to describe a sound in terms of a
	score and an instrument.  MIDI descriptions?
	6. Describe speech in terms of vocal tract? LPC parameters?

12. Editing image sequences as movies:
	You can assemble, insert, dissolve, etc, with image
	sequences the same as you can with sound.  AND you can
	keep the sound synced to the images.
	Movies can be dislayed backward and forwards, at different
	speeds, frame by frame, in short pieces, etc.  They can
	be shown small (micromovies) as movies or arrayed across
	the screen, in order to allow convenient browsing.

13. Databases.
	You can organize all of your images into databases that are
	convenient to browse through.  Images can be tagged with
	descriptors that are used for search: name, date, size, type,
	creator, topic, etc.  The history can also be used for search.
	(so, for instance, you can find all images that involve Lenna
	and 9-tap pyramids.) Perhaps (not likely) we can develop a
	simple similarity metric that allows you to say ``find all the
	images that look kind of like this:''.

14. Data compression.
	Images are stored in a compressed format: a 3-tap pyramid, 4 levels,
	with Huffman coding.  This allows for rapid reconstruction for
	browsing at any size.  Images that must be stored lossless
	can include a residual error image that fills in everything
	that is missing from the compressed image -- even this should
	still take up less space than the original image (?).

15. Object oriented image description.
	To be really object oriented we should distinguish between various
	image descriptions.  A continuous image may be defined by an
	equation, a polynomial, a set of samples if it is bandlimited,
	etc.  It can be rendered into pixel format, but the description
	can exist separately.  Thus, for instance, I can define an image
	as a sum of a ramp and a sinusoid with noise added and a disc
	cut out, without actually computing pixel values.  Later I can
	compute pixels, if I decide to convert it into a pixel-based object.

	This is the approach taken by MacDraw, Postscript, etc.
	It means that some images can have infinite resolution.
  	We can define images as collages of objects, 
	which can later be moved,
	scaled, brought-to-front, etc.,	in MacDraw style.  In this
	approach, ``add'' would be like luminous transparent objects, ``mul''
	like filters, ``replace'' like an opaque object, ``max'' like a
	strange object.  At any moment one can freeze the current rendering as
	a pixel array.

	This approach also includes 3-D models, where the objects are
	solids, polygons, light sources, cameras, etc.

**. Region of Interest (ROI) operations.
	It should be possible to do any operation on a ROI.
	The ROI can be defined by a rectangle, e.g.
	(convolve filt-name im-name :x1 100 :x2 200 :y1 0 :y2 50)
	or, with a mask:
	(convolve filt-name im-name :mask mask-name)
	where mask-name is the name of a binary image mask.

	The result of a ROI operation will be an image of the same
	size as the starting image, but is filled with don't-knows
	(presumably nils) outside the ROI.  Note that this has
	lots of problems (edge handling, etc) and wastes space.
	If we just use rectangular masks, then we won't waste
	space, but will lose the nice spatial correspondence.
	Ask Peter's advice.

**.  Unknown and uncertain values.
	It may often be useful to have pixels take on the value
	``don't know'', or nil.  Examples: you don't bother
	to do an operation outside a ROI; or, you only measure
	velocity along zero crossings; or, you only have disparity
	at a sparse collection of points.

	How to represent uncertainty more generally?  E.G. we
	would like to do Anandan's ellipses of uncertainty, of
	any size and any orientation.


Notes for the Lisp novice:

OBVIUS is written in Lisp, and when you run OBVIUS from the keyboard,
you do so by issuing Lisp commands -- or rather, in the parlance of Lisp,
evaluating Lisp functions.  You don't have to know much
Lisp in order to use OBVIUS for basic image operations, and you can
write your own image processing functions with only a
bare working knowledge of Lisp.

By the way, if you are one of those people who once took a course in
Lisp and was frightened away by CARs, CDRs, and recursion,
please relax.  We're not going to make you use them.
OBVIUS and Common Lisp allow you to do things using
more traditional programming style if you wish.

The following section introduces the basic concepts you will need.

Lisp is interactive: you type an expression, hit <CR>, and Lisp evaluates
the expression.  For instance, to add the integers 1 and 2, type:

LISP> (+ 1 2)
3

Lisp tries to evaluate any expression you give it, and returns the
result -- in this case, 3.

Numbers evaluate to themselves:

LISP> 4
4

A ``list'' is a sequence of things, separated by spaces, 
enclosed within parentheses.
For instance: (a b c), or (1), or (john mary bill) are all lists.
() is also a list, known as the empty list null list.
((1 5) (1 2)) is a list of lists.
A list like (+ (hello?) (33 ``q''))
probably doesn't make any sense, but it is still a list.  

When you hand a list to Lisp for evaluation,
it normally assumes that the first element is a
function name, and that all of the other elements are arguments to the
function.

So,

LISP> (max 1 2 100 4)
100

returns the maximum of the list.

A list that can be evaluated by Lisp is called an ``s-expression'' or
a ``form.'' A form can also be an element in a list.  Thus,
to compute 3 * (4 + 5), you can say:

LISP (* 3 (+ 4 5))
27

Lisp finds the first element of the form, *, and interprets it as a function
call to the multiply function.  It then must evaluate the remaining
elements in the list and so it can multiply them. The 3 evaluates to 3,
but the next element, (+ 4 5), invokes another function call, this time
to the addition function.  It returns 9, so the multiply function 
finishes by evaluating (* 3 9), and returns 27.

Since Lisp is always trying to interpret the first element of a list
as a function name, here's what happens if you just hand it a list of
numbers:

LISP> (77 88 99)
>>Error: Unknown operator 77 in (FUNCTION 77)
EVAL:   Required arg 0 (X): (77 88 99)
:A    Abort to Lisp Top Level
-> 

Lisp thinks you are trying to evaluate a function call ``77'', which
does not exist, so it signals an error and throws you into the
debugger.  Type :A (or whatever your system requires) to get back
to top level.

If you don't want Lisp to evaluate a list, you have to use the single
quote sign to indicate that the list is really just a list.

LISP> '(77 88 99)
(77 88 99)

In this case, Lisp just leaves the list alone and returns it as is.

You can create variables and assign values to them with the function
setq.  To create a variable called ``k'' and assign it the value
9, say:

LISP> (setq  k 9)
9

Now, whenever Lisp evaluates k, it will return 9:

LISP> k
9

(Note that setq doesn't quite behave the way that ordinary Lisp functions are
supposed to, since it doesn't try to evaluate the first argument
following the setq, but accepts it literally instead.  In fact, setq
is short for ``set quote'', and the above expression could be written
as (set 'k 9).  Since it behaves in a non-standard way, setq
is known as a ``special form.'')

Now you can put k in an expression:

LISP> (+ k 5)
14

The value of k is unchanged -- the sum was simply returned.

LISP> k
9

If you want to add 5 to k, and put the result back into k,
then say:

LISP> (setq k (+ k 5))
14
LISP> k
14

[***More to be added: defun, keywords, numeric types...***]
