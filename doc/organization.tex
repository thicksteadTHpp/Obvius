%%%%%%%%%%%%%%%%%%%%%%%%%%%%%%%%%%%%%%%%%%%%%%%%%%%%%%%%%%%%%%%%%%%%%%%%%%%%%%%%%%
\section{Organization of OBVIUS}
\label{sec:organization}

As described in the introduction, OBVIUS is based on three types of
object: viewables, pictures, and panes.  This section covers these
objects in more detail and also explains the memory management system.
We assume that you have gone through the exercizes of the OBVIUS
tutorial.

%%%%%%%%%%%%%%%%%%%%%%%%%%%%%%%%%%%%%%%%%%%%%%%%%%%%%%%%%%%%%%%%%%%%%%%%%%%%%%%%%%

\subsection{Viewables}

\ldef{viewable}s are the objects that one works with in OBVIUS.
In this section, we describe the general
attributes of viewables.  For a discussion of the particulars of each
viewable subclass, see section
\ref{sec:viewables}.  Every viewable (simple or compound) contains the
following slots:\index{{\ptt viewable},slots of}

\begin{itemize}

\item \lsym{name} -- the name of the viewable can be either a string,
a symbol, or nil.  If the name is bound to a symbol, there is a symbol
with that print-name bound to the viewable, i.e., the symbol may be
used to refer to the viewable.  The name of a viewable is usually set
using the standard lisp {\tt setq} function or using the {\tt :->}
(result) keyword argument.  
NOTE: Using (setf slot-value) to set the
name slot will cripple the ability of OBVIUS to keep track of the
existing viewables.  For this reason, the name slot accessor is not
exported, and should not be used.

\item \lsym{superiors-of} -- Some viewables contain other viewables.
These are known as ``compound'' viewables.  For example, a complex
image contains two images (real and imaginary parts).  Viewables
containing other viewables are called {\em compound} viewables.  Each
compound viewable knows how to access its ``inferior'' viewables, and
each of the inferior viewables has a pointer to the ``superior''
(compound) viewable.  This slot contains a list of all viewables
containing the viewable as a sub-viewable.  For instance, if the
viewable is the real-part of a complex-image, the list will contain
that complex-image.

\item \lsym{display-type} -- this slot contains the name of the 
sub-class of picture \index{{\ptt picture}} that will be used when the viewable is
automatically displayed.  A nil value indicates that the viewable
should {\em not} be displayed automatically.  The slot may be set using the
standard setf syntax: {\tt (setf (display-type \abox{viewable})
'\abox{picture-type})}

\item  \lsym{pictures-of} -- this slot contains a list of all of the
pictures of the viewable.  There may be several different types of
picture of the same viewable, and there may be more than one of a
given picture type.  For example, there may be a grayscale picture
{\em and} a surface plot of an image, or there may be two different
grayscale pictures.

\item \lsym{history} -- this slot contains a history object for the
viewable.  See the section below for more information.

\item \lsym{current} -- this slot contains a number that is incremented
each time a viewable is modified.  It is used to keep track of whether
the pictures of that viewable are up-to-date or not.  There is a
similar slot in the picture class.

\item \lsym{info-list} -- this slot contains a lisp ``plist'' (property
list) of attributes.  For example, if the mean of an image is
computed, it is stored in this p-list so it does not need to be
recomputed.  The same is true for histograms, minimum, maximum, and
other image statistics.  The info-list is set to nil when the viewable
is destructively modified (this is done by the {\tt set-not-current} method).
\end{itemize}

The following functions may be used to access the info-list of a viewable:
\begin{itemize}
\item\lfun{info-list}{ viewable}
Returns the info-list of an image.

\item\lfun{info-get}{ viewable slot}
Returns the value of the slot named {\em slot} in im's info-list.  May be
used with setf to set the value instead of using info-set.

\item\lfun{info-set}{ viewable slot value}
Sets the value of the slot named {\em slot} in im's info-list to have
the value {\em value}.  You can also use {\tt (setf (info-get vbl
slot) val)}.

\end{itemize}

\mysubsubsec{Displaying Viewables}

\index{{\ptt viewable},auto display}
When a viewable is returned to the top-level lisp listener, it will be
displayed automatically if the global parameter
\lsym{*auto-display-viewables*} is non-nil.  If a picture 
\index{{\ptt picture}} of the viewable
already exists, it will be brought to the top of the stack of the
current pane \index{{\ptt pane}} if the variable
\lsym{*preserve-picture-pane*} is nil.  If this variable is non-nil,
then the picture will stay on the same pane, but still be brought to
the top of the stack.  Otherwise, a picture is created of the type
defined by the display-type slot of the viewable and pushed onto the
picture stack of the current pane.  If you wish to explicitly create a
picture of a viewable with a particular set of display parameters, use
the
\lsym{display} function (described in Section \ref{sec:pictures}).
Section \ref{sec:pictures} also describes methods for altering the
parameters of pictures.

\mysubsubsec{Modifying Viewables}

\index{{\ptt viewable},destructive modification}
When a viewable is destuctively modified (e.g. by passing it as the
\lsym{:->} argument to a function) the method \lsym{set-not-current} is
called.  This method tells OBVIUS that the pictures that depend on
that viewable need to be recomputed.  In general, the method does
nothing to the viewable, but calls itself on each of the superiors of
the viewable, and on each of the pictures of the viewable.  The
pictures then either update themselves immediately, or indicate to the
user that they are ``not current'' by placing a double asterisk in the
title bar (this is consistent with the double asterisk Emacs uses to
indicate a buffer has been modified).  The display may be updated by
refreshing the pane via a mouse click (see Section
\ref{sec:mouse}).  Note that modifying a viewable does {\em not} cause
related viewables to be re-computed.  For example, if a histogram is
computed from an image and the image is subsequently modified, the
histogram will not be recomputed.

\mysubsubsec{Destroying Viewables}

A viewable may be destroyed by evaluating the expression {\tt (\lsym{destroy}
\abox{viewable})} or via a mouse click (see Section
\ref{sec:mouse}).  The function \lsym{purge!} destroys all viewables.
When a viewable is destroyed, the following things occur:
\begin{enumerate}
\item Superiors of the viewable are notified.  Note that some
superiors will not allow their inferiors to be destroyed (e.g. a
complex image will trigger a continuable error when the user tries to
destroy its real part).
 
\item Inferiors of the viewable are notified.  Typically, the
inferiors just take the superior off of their {\tt superiors-of}
lists.  If the global parameter
\lsym{*auto-destroy-orphans*} is non-nil, the inferior is destroyed
(see section on orphaned viewables below).  

\item All pictures of the viewable are destroyed (i.e., the {\tt
destroy} method is called on each picture).  The static space used by
the pictures is freed (see section \ref{sec:memory}).

\item The static space of the viewable is freed (see Section
\ref{sec:memory}).

\item If the viewable name slot contains a symbol, it is unbound in
the current package of the global Lisp environment.

\item The name of the viewable is set to the keyword
\lsym{:destroyed}, as an indicator to the user in case they try to
re-use the viewable.
\end{enumerate}

\mysubsubsec{Orphans}
\index{orphans}
A viewable is called an ``orphan'' if it is no longer accessible.  As
far as OBVIUS is concerned, a viewable is considered inaccessible if
it has no superiors, there are no pictures of it, and there is no
global symbol bound to it.  When you run the OBVIUS garbage collector,
\lsym{ogc}, it will recycle all orphaned viewables.  Furthermore, if
the global variable \lsym{*auto-destroy-orphans*} has a non-nil value,
OBVIUS will attempt to destroy orphans when they are created.  Thus,
users of OBVIUS should understand when they are creating orphans!

There are three situations when OBVIUS can prevent the ``orphan'ing'' of
an existing viewable::
\begin{enumerate}
\item The (symbol) name of the viewable is (re-)bound (using {\tt setq} or
{\tt set}) to another value.
\item The only picture of the viewable is popped from a pane (destroyed).
\item The viewable is an inferior of a compound viewable, and that
compound viewble is destroyed.
\end{enumerate}
In each of these cases, if the global parameter
\lsym{*auto-destroy-orphans*} is non-nil, the orphan will be
automatically destroyed.  

{\em Note that this can be dangerous.} For example, if a user writes
code that returns an inferior viewable from within a function, but the
function destroys the superior, the inferior may be destroyed {\em
before} it is returned to top level.  Note also that automatic
destruction of orphans {\em will not} prevent the user from creating
new viewables that are orphans.  For example, one can create a list of
images with no name and avoid dislaying them.  These images will be
orphans and will not be preserved by the OBVIUS garbage-collector
({\tt ogc}), although the user will have access to them.  If you want
to return several images from a function, you should return them as an
image-sequence, or using the Common Lisp multiple-values facility.
See section~\ref{sec:hacking} for a more thorough discussion of this
problem.

\mysubsubsec{History and Constructor}
The \lsym{constructor} function produces a lisp S-expression that
would evaluate to a copy of the viewable.  When you save a viewable to
a datfile (see section~\ref{sec:datfile}, this expression is written
to the descriptor file.  It is also useful as a reminder of where the
viewable came from.  The function {\tt (\lsym{history-sexp} <viewable>
\&key (level 1))} allows the user to print a condensed listing of the
history, substituting symbols for the viewables with symbol name
slots.  The {\tt :level} parameter determines how many nested levels
of history (function calls) will be displayed before the system
attempts to replace them with symbols.  These functions are defined in
the file {\bf viewable.lisp}.


%%%%%%%%%%%%%%%%%%%%%%%%%%%%%%%%%%%%%%%%%%%%%%%%%%%%%%%%%%%%%%%%%%%%%%%%%%%%%%%%%%

\subsection{Pictures and Panes}
\label{sec:pictures}

A \ldef{pane} is a window that contains a stack of pictures.  The
picture on top of the stack is the one that is displayed on the
screen.  Each picture is allowed to reside on the stack of only one
pane.  To create a new pane, the user should call the function 
\lsym{new-pane}. The keyword arguments :left, :bottom, :right, and :top
specify the position in pixels of the lower left and upper right
corners of the pane.

The system maintains a notion of the {\em selected} or {\em current}
pane.  In general, clicking the mouse in a pane will select that pane
(see section \ref{sec:mouse}).  Whenever a viewable is automatically
displayed by the system, it is displayed in the current pane.

A \ldef{picture} is a particular visual representation of a viewable.

\mysubsubsec{Picture Slots and Default Values}
The following slots are found in every picture type:
\begin{itemize}
\item {\bf pane-of} -- the pane in which this picture lives.  Pictures
cannot exist in isolation; they must be on a picture stack.

\item {\bf viewable} -- the viewable of which this is a picture.

\item {\bf system-dependent-frob} -- a window-system dependent
intermediate representation.  The window system knows how to rapidly
``render'' the ``frob'' into a pane.  For gray pictures on X-screens,
the frob is an offscreen pixmap.  For graphs with retain-bitmap
non-nil, it is an offscreen bitmap.  But for graphs with retain-bitmap
nil, it is an object containing lists of graphical objects (such as
lines) to draw.

\item {\bf index} -- a unique registration number identifying the picture.

\item {\bf current} -- a number that is is used to
keep track of whether the picture is up-to-date with its viewable
(given that the viewable may have been modified since the picture was
originally made).  The current slot is reset each time the picture is
updated.  There is a similar slot for the viewable class.

\item {\bf zoom} -- magnification of the picture, relative to its
``natural'' size (this is described in detail below).

\item {\bf overlays} -- this slot contains a list of the viewables that
are overlayed on this picture (see section \ref{sec:overlays}).

\item {\bf y-offset} and {\bf x-offset} -- these slots specify the
position of the picture within the pane (relative to centered position).

\end{itemize}

In addition to those listed above, each picture class has parameters
that control its computation from the underlying viewable.  For
example gray pictures have \lsym{scale} and \lsym{pedestal} parameters
that specify the transformation of the floating-point image array into
a grayscale colormap.  The values of these parameters may be changed
using the {\tt Current Picture} dialog from the {\tt Parameters} menu
on the contral panel, or using the {\tt M-left} mouse click.  The
parameter values may also be reset using the
\lsym{setp} macro, which acts on the picture on top of the current
pane.  The {\tt setp} macro is described in detail below.

When a new picture is created, default values are chosen for the
picture parameters.  The default slot values for each picture class
may be set from the {\tt Picture Defaults} sub-menu of the {\tt
Parameters} menu.  Typically, the default value may be a number or it
may be the keyword{\tt :auto}.  If it is a number, that number will be
used as an initial slot value for all pictures of this type.  If the
default value is not a number, then OBVIUS will choose parameter
values automatically.

For details about the parameters for each picture class, see the
descriptions in Section \ref{sec:viewables}.

\mysubsubsec{Manipulating Pictures and Panes}

In general, the user should not need to directly manipulate pictures
or panes.  OBVIUS is designed with the intention that the user perform
computations on {\em viewables}.  There are, however, several methods
provided that allow the user to alter pictures:
\begin{description}
\item\lsym{*current-pane*} \\
This global symbol is bound to the current pane.  This symbol can be
used when calling functions that take a pane as an argument.

\item\lfun{display}{ \&optional display-type \&key pane make-new \&allow-other-keys}
This function is used to create a picture of a viewable.  The optional
{\tt display-type} argument should be a symbol corresponding to the
name of a picture subclass (eg. {\tt 'gray}, or {\tt 'vector-field}).
The default value is t, meaning to use the \lsym{display-type} slot of
the viewable.  The {\tt :pane} keyword argument specifies the pane in
which to display the picture and defaults to the value of
\lsym{*current-pane*}.  If this is nil, a new pane is created.  The
{\tt display} function first tries to find an existing picture of the
desired type.  If no picture of the right type exists, or if a non-nil
value is passed with the keyword \ldef{:make-new}, a new picture is
created with all default parameter values.  Otherwise, one of the
existing pictures is displayed, and updated if it is out-of-date and
\lsym{*auto-update-pictures*} is non-nil.  
All additional keyword args ({\tt picture-initargs}) are used as
initialization arguments in creating the picture (i.e., they are
passed along to {\tt make-instance}).  They typically correspond to
initargs for the slots of the picture being created.  For example, you
can create a gray picture of an image, with the scale factor set to
1.0, and the zoom factor set to fill the pane, as follows:
\begin{verbatim}
(display <image> 'gray :scale 1.0 :zoom :auto)
\end{verbatim}

\item\lfun{refresh}{ \&optional pane}
This function re-displays the picture on top of the specified pane.
The optional {\tt pane} argument defaults to the value of
\lsym{*current-pane*}.  This function is typically called via a mouse
click (see Section \ref{sec:mouse}).

\item\lfun{next-pane}{}
This function sets the current pane to be the next pane in the list of panes.

\item\lfun{cycle-pane}{ \&optional pane number}
This function cycles the circular picture stack of a pane and
refreshes the display.  The default pane is the value of
\lsym{*current-pane*}.  The optional {\tt number} argument defaults to
a value of 1, and specifies how many pictures to cycle past.  Negative
values cycle in reverse. This function is typically called via a mouse
click (see Section \ref{sec:mouse}).

\item\lfun{current-viewable}{}
Returns the viewable currently visible in the current pane.

\item\lfun{get-viewable}{ \&optional pane}
Returns an S-expression that will evaluate to the viewable currently
visible on top of the pane.  Optional argument {\tt pane} defaults to the value of
\lsym{*current-pane*}.   This function is typically called via a mouse
click (see Section \ref{sec:mouse}).

\item\lfun{pop-picture}{ \&optional pane}
This function removes the picture on the top of a picture stack and
destroys it.  The underlying viewable, however, is left intact.  The
default pane is the current pane.  It is usually called via a mouse
click (see Section \ref{sec:mouse}).

\item\lfun{setp}{ \&rest args}
It is often desirable to modify the parameters that are associated
with the various picture types.  The macro setp has been provided for
this purpose.  The syntax for setp is just like setq except that all
of the symbols to be set should correspond to slot names in the
picture on top of the current pane.  After setting the new slot
values, the picture will be recomputed using the new values and the
display will be refreshed.  Note that some picture slots are not
settable; calling setp on these will have no effect.  Picture
parameters can also be reset using the {\tt Current Picture} option
from the {\tt Parameters} menu.

\item\lfun{getp}{ \&optional slot-name}
This macro is complementary to setp.  It takes a single argument that
must be the name of a slot in the current picture and returns the
value of that slot.  When called with no arguments, it prints a list
of the names and values of {\em all} of the slots of the current
picture.  
\end{description}

\mysubsubsec{Zoom}

Every picture class has a {\tt zoom}\index{{\ptt zoom},slot} parameter that
controls the size of the picture in the pane.  The zoom amount is
relative to the ``natural'' size of the picture (this depends on the
type of picture).  The {\tt zoom} parameter may be set to {\tt
:auto}\index{{\ptt :auto},zoom}, a number, or a list of two numbers.  If it
is {\tt :auto}, then OBVIUS sizes the picture to fill the pane.  If it
is a list, then OBVIUS sizes the picture to have dimensions equal to
that list.  If {\tt zoom} is a number, then the size of the picture
will be this multiple of its natural size.  For example, if a 64x64
image is displayed as a gray picture with zoom = 2, then the picture
will by 128x128 pixels.  If zoom = 1/2, then the picture will be 32x32
pixels.  For some pictures (e.g., gray), zoom will be rounded to the
nearest integer, or reciprocal of an integer.  For others (e.g.
graph), it can take a floating point value.

The value of the zoom parameter may be changed from the {\tt Current
Picture Parameters} menu, using the {\tt setp} macro, or by using the mouse
(see section~\ref{sec:mouse}).

% The default value for {\tt zoom} may be set separately for each
% picture class from the {\tt Picture Defaults} submenu of the {\tt
% Parameters} menu.  If the default is {\tt nil}, then OBVIUS chooses a
% value automatically each time a new picture is made.  Typically, the
% automatic choice is {\tt zoom} = 1.

%%%%%%%%%%%%%%%%%%%%%%%%%%%%%%%%%%%%%%%%%%%%%%%%%%%%%%%%%%%%%%%%%%%%%%%%%%%%%

\subsection{File System Interface}
\label{sec:datfile}

OBVIUS uses a subset of the {\em datfile}\index{datfile} image file format used by
several groups at the Media Lab and in other labs at MIT.  A datfile
is a directory containing one or more files.  One of these files,
named ``data'', contains the data organized in a rectangular matrix
with any number of dimensions.  A separate file in the datfile
directory, named ``descriptor'', contains a pseudo-LISP description of
the data file.  Keyword/value pairs of ASCII strings, enclosed in
matching parentheses, describe the organization, data type, history
and other aspects of the datfile.  The advantage of this scheme is
that information about the image is kept separate from the image data
itself, and is human-readable and easily modified.  No special-format
headers are required.

A typical descriptor file looks like the following (the corresponding
data file is 16K bytes):
\begin{verbatim}
(_data_type unsigned_1) 
(_data_sets 1) 
(_channels 1) 
(_dimensions 256 256) 
\end{verbatim}
The most important keywords are {\tt \_data\_type} and {\tt
\_dimensions}.  The {\tt \_dimensions} keyword specifies the size of the
image (y-dimension is first).  The {\tt \_data\_type} keyword
in the descriptor file specifies the type of the data file elements,
and may take any of the following values (on the left is the
{\tt \_data\_type} specification in the descriptor file, and on the right is
the Common Lisp type specifier):
\begin{itemize}
\item unsigned\_1 = '(unsigned-byte 8)
\item unsigned\_4 = '(unsigned-byte 32)
\item float = 'single-float
\item ascii = 'character
\end{itemize}
The {\tt \_data\_sets} and {\tt \_channels} keywords are currently not
used by OBVIUS, but they must both be specified in the descriptor file
and equal to $1$.  In the future, {\tt \_channels} may be used for color
images (3 channels in an rgb image stored consecutively in the same
data file) and {\tt \_data\_sets} will be used for image sequences (multiple
images from an image sequence stored in the same data file).  

Currently, OBVIUS stores {\tt image-sequences} as a datfile directory
with one descriptor file and several data files named ``data0'',
``data1'', etc.  The keyword {\tt \_data\_files} in the descriptor
file specifies how many data files are present.

The function \lsym{load-image} (see Section \ref{sec:operations}) loads
images from datfiles and the function \lsym{save-image} saves images to
datfiles.  The {\tt load-image} function will also load images from a
file that is a stream of bytes in row-major order with one byte per
pixel.

OBVIUS is designed to load images in various machine formats.  Some
machines (e.g., Suns and Symbolics) store 32-bit numbers in reversed
byte order from each other.  OBVIUS uses the {\tt :byte-order}
keyword in the descriptor of a datfile to determine the byte-order of
the file.  If necessary, {\tt load-image} will reverse the order of
the bytes as it reads in a 32bit image file.  OBVIUS is not currently
set up to load different floating point formats.  Rather, you should
use ascii or 32-bit file format if you have different machines at your
site.

The following describes the functions for load and saving images and
image-sequences:
\begin{description}
\item\lfun{load-image}{ path \&key ysize xsize reverse-bytes \res}
If path is a directory, load-image assumes that path is a datfile, and
{\tt :ysize} and {\tt :xsize} are ignored.  Otherwise it assumes that
path is an image file stored in row-major order with one byte per
pixel.  In such a case the keywords {\tt :ysize} and {\tt :xsize} may
be used to specify the size of the image.  If they are not specified
then load-image assumes that the image is square.  For a datfile,
OBVIUS uses the information in the descriptor file to determine the
element-type, the byte-order, and the dimensions of the data.

First, {\tt load-image} converts the incoming data into floats.  Then
it uses a scale and pedestal, if specified in the descriptor file, to
rescale the image data.

The {\tt :reverse-bytes} argument is only relevant for 32-bit
datfiles.  If {\tt :reverse-bytes} is nil then OBVIUS uses the
byte-order keyword in the descriptor file to determine whether or not
to reverse the bytes.  If the {\tt :reverse-bytes} is non-nil, then
load-image ignores always reverses the bytes (ignoring the byte-order
keyword in the descriptor file).

If :\res is not specified and the global {\tt
*auto-bind-loaded-images*} is non-nil, then the image will be named
(see Section \ref{sec:viewables}) with the file-name or directory-name
of path (not the whole path-name).  See the tutorial for a simple
example.

The {\tt load-image} function calls the {\tt load-image-sequence}
function if the {\tt \_data\_files} keyword is present in the
descriptor file.

\item\lmeth{save-image}{image}{ image path \&key element-type auto-scale reverse-bytes df-keys}
Saves image as a datfile in path.  The {\tt :element-type} keyword specifies
what type of file to save it as.  {\tt :Element-type} may be '(unsigned-byte
8), '(unsigned-byte 32), `single-float, or 'character.  The default is
'(unsigned-byte 8).  WARNING: Floating point word formats are
different for different manufacturers, so it may not be possible to
load an image that was saved from a different machine (use 'character
or '(unsigned=byte 32) instead of 'float). 

If {\tt :element-type} is '(unsigned-byte 32) then the image is
rescaled to fill the 32bit range.  The appropriate scale and pedestal
are written to the descriptor file so that the 32-bit numbers will be
rescaled correctly when the image is reloaded.  If the reverse-bytes
keyword is T (default is nil) and the {\tt :element-type} is
'(unsigned-byte 32), then the byte-order is swapped during write.

If {\tt :element-type} is '(unsigned-byte 8) and {\tt :auto-scale} is
NIL then no rescaling is done.  If {\tt :auto-scale} is T (the
default) then the image is rescaled to fill the range from 0 to 255,
and the appropriate scale and pedestal are written to the descriptor
file so that the 8-bit numbers will be rescaled correctly when the
image is reloaded.

\item\lfun{load-image-sequence}{ path \&key start-index end-index \res}
Loads an image sequence from a datfile directory with multiple data
files name ``data0'', ``data1'', etc.  The keywords :start-index and
:end-index may be used to load a subsequence.  The load-image function
calls the load-image-sequence function if the \_data\_files keyword is
present in the descriptor file.

\item\lmeth{save-image}{image-sequence}{ sequence path \&key element-type auto-scale reverse-bytes df-keys start-index end-index}
Saves a sequence as a datfile directory with multiple data files.  The
keywords are the same as for saving images.  The :start-index and
:end-index keywords may be used to save a subsequence.
\end{description}

%%%%%%%%%%%%%%%%%%%%%%%%%%%%%%%%%%%%%%%%%%%%%%%%%%%%%%%%%%%%%%%%%%%%%%%%%%%%%%%%%%

\subsection{Memory Management}
\label{sec:memory}

Lisp memory space is divided into several different areas.  The most
important distinction is between the dynamic and static areas.  All
user-defined storage space normally resides in the ``dynamic'' areas.
Lucid Lisp provides several functions for examining and manipulating
the dynamic Lisp memory space from top level.  A complete explanation
is available in the Lucid Common Lisp User's Guide.  A few very useful
functions are:
\begin{description}
\item\lfun{change-memory-management}{ \&key growth-limit expand ...}
If a number is supplied for the {\tt :growth-limit} keyword, it specifies
the maximum Lisp process size in 64k segments.  If a number is
supplied for the {\tt :expand} keyword, the Lisp dynamic space is expanded
by that many 64k segments.

\item\lfun{room}{ \&optional info}
This function prints out information about the current state of the
lisp memory map.  If the optional argument {\tt info} is t, more
information is given.
\end{description}

In order to avoid garbage collecting large arrays such as image
arrays, OBVIUS allocates storage for viewables and pictures from the
static heap.  {\em Since these arrays are never garbage collected,
they must be returned explicitly to the system}.  OBVIUS maintains
separate ``heaps'' of static storage space for different array types
(e.g., 'single-float, 'bit, etc).  The memory allocator is designed to
create heaps for any element-type that is a valid argument to the
Common Lisp {\tt make-array} function.  Arrays are allocated from and
returned to the static heaps with the functions {\tt allocate-array}
and {\tt free-array} as described in Section~\ref{sec:hacking}.

Sometimes in the midst of a computation, OBVIUS will need to allocate
an array from one of the static heaps that has no more free space.
The global parameter \lsym{*auto-expand-heap*} controls the behavior in
this situation.  If {\tt *auto-expand-heap*} is non-nil, then OBVIUS
will automatically call the {\tt expand-heap} function (described
below) to increase the size of the appropriate heap.  The amount of expansion is
controlled by the variable \lsym{*heap-growth-rate*}, which should
contain an integer specifying the number of {\em elements} to add to
the heap.  If {\tt *auto-expand-heap*} is nil, then OBVIUS will signal
a continuable error: you can either abort the computation or enter
{\tt :c} to expand the heap by {\tt *heap-growth-rate*} elements.

Several functions are provided for top-level management of the static
heaps:
\begin{description}

\item\lfun{expand-heap}{ type \&optional size}
This function expands the size of OBVIUS's static array area, for the
given type of element.  The type argument can be any array type.
Examples are: 'single-float, '(unsigned-byte 8), and 'bit.  The
optional parameter {\tt size} specifies the number of elements by
which to expand the respective space.  It defaults to the value of the
global parameter {\tt *heap-growth-rate*}.

\item\lfun{heap-status}{}
Prints a description of the current static memory usage.

\item\lfun{ogc}{ \&key verbose}
A primitive garbage collector that offers a means of reclaiming lost
storage space (i.e., allocated arrays that belong to orphaned \index{orphans}
viewables -- see the previous section on orphans for an explanation).
Note that unlike the Common Lisp garbage collector, {\tt ogc} must be
called manually by the user.  {\tt Ogc} reclaims static arrays that do
not belong to: (1) pictures, (2) viewables of pictures, (3) viewables
that are bound to symbols in the current package or the {\tt USER}
package, or (4) inferior or superior viewables of the aformentioned
viewables.  All arrays that are not preserved are returned to the heap
for reallocation.  {\tt Ogc} also prints a list of symbols in the
current package that are currently bound to viewables, so that the
user may explicitly destroy these.  The former name for this
function was \lsym{scrounge!}, which has been retained for
back-compatibility.

{\bf Warnings:} this function should {\em only} be called from top
level, since it will not preserve viewables bound to local variables.
Furthermore, it only preserves symbols that are in the current package
(the name of the current package is the value of the variable {\tt
*package*}) or the {\tt USER} package.  Furthermore, if you have
created viewables that are not accessible through pictures, symbols,
or inferiors or superiors, {\em these viewables will be recycled by
the garbage collector}. 

\end{description}


